\setmainfont{Times New Roman}
\setCJKmainfont{SimSun}

%封面字段,使用recomand重定义
\newcommand\thesistitle{(本模板所有格式、字体字号均已设置完毕,无须更改      黑体小二号字)}
\newcommand\academy{(黑体三号字)}
\newcommand\stuname{(黑体三号字)}
\newcommand\grade{(黑体三号字)}
\newcommand\subject{(黑体三号字)}
\newcommand\stunumber{(黑体三号字)}
\newcommand\mentor{(黑体三号字)}
\newcommand\thesisdate{2025年\hspace{1em}2月\hspace{1em}3日}

\geometry{
	a4paper,
	left=23mm,
	right=28mm,
	top=28mm,
	bottom=23mm
}

%设置参考问格式
\setlength{\biblabelsep}{0 mm}
\renewcommand{\baselinestretch}{1.5}
\AtBeginBibliography{
	\fontsize{10.5pt}{15.75pt}\rmfamily
}

%设置页眉页脚
\pagestyle{fancy}
\setlength{\topmargin}{0pt}
\setlength{\voffset}{-1cm}
\setlength{\hoffset}{0.5cm}
\setlength{\footskip}{0pt}
%正文前去除页眉横线,清空页脚
\renewcommand{\headrulewidth}{0pt}
\fancyhead[C]{}
\fancyhead[L]{}
\fancyhead[R]{}
\fancyfoot[C]{}
\renewcommand{\headrulewidth}{0pt}
\newcommand{\sanhao}{\fontsize{16pt}{24}\selectfont}

\usepackage{pifont}

%Word在章节编号里用的分隔符是方形点,自己定义一个
\newcommand\mydot{\hspace{0.1em}{\heiti\fontsize{2pt}{0}\ding{110}}\hspace{0.3em}}




%定义目录样式
\renewcommand{\contentsname}{\heiti\zihao{-2}目录}
% 定义章节条目格式
\titlecontents{chapter}% 作用于章级别
[0em]% 左边距
{\vspace*{1cm}\bfseries}% 章节前的垂直空间和字体加粗
{\thecontentslabel\enspace}% 编号后的格式
{}% 对于没有编号的部分
{\hfill\contentspage}[\vspace*{2mm}]% 页码前的填充和章节后垂直空间

% 定义节条目格式
\titlecontents{section}% 作用于节级别
[0em]% 左边距
{\heiti\zihao{-4}}% 章节前的垂直空间和字体加粗
{\thecontentslabel\mydot\enspace}% 编号后的格式
{}% 对于没有编号的部分
{\normalfont\titlerule*[3pt]{.}\zihao{-4}\hspace{-0.8em}\contentspage}% 使用点线连接标题和页码

% 定义小节条目格式
\titlecontents{subsection}% 作用于小节级别
[2em]% 缩进量,相对于节的额外缩进
{\normalfont}% 字体设置
{\thecontentslabel\enspace}% 编号后的格式
{}% 对于没有编号的部分
{\normalfont\titlerule*[3pt]{.}\zihao{-4}\hspace{-0.8em}\contentspage}% 使用点线连接标题和页码

\titlecontents{subsubsection}% 作用于小节级别
[4em]% 缩进量,相对于节的额外缩进
{\normalfont}% 字体设置
{\thecontentslabel\enspace}% 编号后的格式
{}% 对于没有编号的部分
{\normalfont\titlerule*[3pt]{.}\zihao{-4}\hspace{-0.8em}\contentspage}% 使用点线连接标题和页码

\newcommand{\specialsectionwithoutbold}[1]{%
	\titlecontents{section}[0em] % 设置特定章节的格式
	{\heiti\zihao{-4}}
	{\thecontentslabel.\enspace}% 编号后的格式
	{}% 对于没有编号的部分
	{\normalfont\titlerule*[3pt]{.}\zihao{-4}\hspace{-0.8em}\contentspage}% 使用点线连接标题和页码

	\section*{#1}
	\addcontentsline{toc}{section}{\newline #1}

	\titlecontents{section}% 作用于节级别
	[0em]% 左边距
	{\heiti\zihao{-4}}% 章节前的垂直空间和字体加粗
	{\thecontentslabel\mydot\enspace}% 编号后的格式
	{}% 对于没有编号的部分
	{\normalfont\titlerule*[3pt]{.}\zihao{-4}\hspace{-0.8em}\contentspage}% 使用点线连接标题和页码

}

% 设置 section 编号格式
\renewcommand{\thesection}{\arabic{section}}

% 设置 subsection 编号格式
\renewcommand{\thesubsection}{\thesection\mydot\arabic{subsection}}

% 设置 subsubsection 编号格式
\renewcommand{\thesubsubsection}{\thesubsection\mydot\arabic{subsubsection}}

\titleformat{\section} % 作用于 section 级别
[block] % 标题布局,可选 hang, block, display 等
{\centering\heiti\zihao{-2}} % 格式化命令(如字体系列和大小)
{\heiti\thesection\mydot} % 标签格式
{0.5em} % 标签与标题文本之间的水平距离
{\indent} % 在标签之后、标题文本之前执行的代码

% 调整 section 的前后间距
\titlespacing*{\section} % 作用于 section 级别
{2em} % 左边距
{0.5em} % 标题前的垂直空间
{1em} % 标题后的垂直空间

\titleformat{\subsection} % 作用于 section 级别
[block] % 标题布局,可选 hang, block, display 等
{\heiti\zihao{-3}} % 格式化命令(如字体系列和大小)
{\heiti\thesubsection } % 标签格式
{0.5em} % 标签与标题文本之间的水平距离
{\indent} % 在标签之后、标题文本之前执行的代码

% 调整 section 的前后间距
\titlespacing*{\subsection} % 作用于 section 级别
{2em} % 左边距
{0.5em} % 标题前的垂直空间
{1em} % 标题后的垂直空间

\titleformat{\subsubsection} % 作用于 section 级别
[block] % 标题布局,可选 hang, block, display 等
{\heiti\zihao{-4}} % 格式化命令(如字体系列和大小)
{\heiti\thesubsubsection} % 标签格式
{0.5em} % 标签与标题文本之间的水平距离
{\indent} % 在标签之后、标题文本之前执行的代码

% 调整 section 的前后间距
\titlespacing*{\subsubsection} % 作用于 section 级别
{2em} % 左边距
{0.5em} % 标题前的垂直空间
{1em} % 标题后的垂直空间




%封面
%word与latex算法不同,具体数值仅供参考,以形似为标准(看着像)
\newcommand{\coverpage}{
%封面图
	\begin{figure}[!h]
		\centering
		\resizebox{10.88cm}{3.15cm}{\includegraphics{cover_image.jpg}}
	\end{figure}

	\begin{center}
		\heiti\fontsize{28}{42}\ziju{0.22}{本科生毕业论文}
	\end{center}
	\vspace{30pt}
	\setlength{\cmidrulewidth}{0.5pt}
%封面表
	\begin{table}[h]
		\hspace{0.3em}
		\renewcommand{\baselinestretch}{1.4}
		\renewcommand{\arraystretch}{1.82}
		{
			\begin{tabularx}{\textwidth} {
					>{\arraybackslash\hsize=3.5cm}X
					>{\arraybackslash\hsize=10.5cm}X
				}
				\raisebox{-0pt}{\noindent\arraybackslash\heiti\zihao{-2}{论文题目:}} & \noindent\arraybackslash\heiti\zihao{-2}{\thesistitle}                                      \\
				\cmidrule{2-2}
				\noindent\arraybackslash\heiti\zihao{3}{学\hspace{2em}院:}         & \noindent\arraybackslash\heiti\zihao{3}{\academy}                                           \\
				\cmidrule{2-2}
				\noindent\arraybackslash\heiti\zihao{3}{年\hspace{2em}级:}         & \noindent\arraybackslash\fontspec[BoldFont={SimHei}]{SimHei}\heiti\sanhao{\grade}           \\
				\cmidrule{2-2}
				\noindent\arraybackslash\heiti\zihao{3}{专\hspace{2em}业:}         & \noindent\arraybackslash\heiti\zihao{3}{\subject}                                           \\
				\cmidrule{2-2}
				\noindent\arraybackslash\heiti\zihao{3}{姓\hspace{2em}名:}         & \noindent\arraybackslash\heiti\zihao{3}{\stuname}                                           \\
				\cmidrule{2-2}
				\noindent\arraybackslash\heiti\zihao{3}{学\hspace{2em}号:}         & \noindent\arraybackslash\heiti\fontspec[BoldFont={SimHei}]{SimHei}\heiti\sanhao{\stunumber} \\
				\cmidrule{2-2}
				\noindent\arraybackslash\heiti\zihao{3}{指导教师:}                   & \noindent\arraybackslash\heiti\zihao{3}{\mentor}                                            \\
				\cmidrule{2-2}
			\end{tabularx}
		}
	\end{table}

	\vspace{42 pt}
%日期:年月日
	\begin{center}
		\fontspec[BoldFont={SimHei}]{SimHei}\heiti\sanhao{\thesisdate}
	\end{center}
	\newpage
	\songti\zihao{-4}
}

